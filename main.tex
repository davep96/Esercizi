\documentclass[10pt, a4paper]{article}
\usepackage[T1]{fontenc}
\usepackage[utf8]{inputenc}
\usepackage[italian]{babel}
\usepackage{subfiles}

\usepackage{youngtab}
\usepackage{physics}
\usepackage{tikz}
\usepackage{amsmath}
\usepackage{amssymb}
\usepackage{amsthm}
\usepackage{siunitx}
\newcommand*\chem[1]{\ensuremath{\mathrm{#1}}}
\usepackage[margin=1.00in]{geometry}
\usepackage{lmodern}

\theoremstyle{plain} 
\newtheorem{ese}{Esercizio}

\newenvironment{svol}{\paragraph{Svolgimento:}}{\hfill$\square$\newline}

\newenvironment{dati}{\paragraph{Dati:}}{\hfill}

\begin{titlepage}
	\title{Esercizi del corso di Istituzioni di Fisica Nucleare e Subnucleare}
	\author{Davide Passaro}
	\date{Anno scolastico 2017-2018}
\end{titlepage}

\begin{document}
	\maketitle
	\section{Fisica nucleare}
		\subfile{SubFiles/Radioattività.tex}
		\newpage
		\subfile{SubFiles/DecadimentiAlpha.tex}
		\newpage
		\subfile{SubFiles/Fissione.tex}
		\newpage
		\subfile{SubFiles/ShellModel.tex}
		\newpage
	\section{Fisica delle particelle}

		\subfile{SubFiles/NuclearFusion.tex}
		\newpage
		\subfile{SubFiles/EtaDecay.tex}
\end{document}