\documentclass[../main]{subfiles}

\begin{document}
	\begin{ese}[3.2]\label{ese:UDecay}
		L'abbondanza naturale di $ ^{235}\chem{U} $ è dello $ 0.7\% $ di $ ^{238}U $. Assumendo che i processo di nucleosintesi producano approssimaticamente le stesse quantità di $ ^{235}U $ e di $ ^{238}U $, quanto è \emph{vecchio} l'uranio presente sulla terra. Si ricordano i dati delle due catene della radiattività naturale:
		\begin{gather}
		A=4n+2,\ ^{238}\chem{U},\ \tau_{\frac{1}{2}}=\SI{4.5e9}{yr} \\
		A=4n+3,\ ^{235}\chem{U},\ \tau_{\frac{1}{2}}=\SI{7.15e8}{yr}
		\end{gather}
	\end{ese}
	\begin{svol}
		Osservando solamente i dati si può osservare che il tempo di dimezzamento dell'uranio 235 è minore di quello 238 di quasi un ordine di grandezza. Questa osservazione conferma il fatto che, avendo uguali quantità di uranio nei suoi due isotopi, col passare del tempo, il rapporto della quantità di Uranio 235 con quella di Uranio 238 andrà diventerà man mano sempre minore di uno. La strategia di soluzione di questo problema sarà dunque quella di analizzare l'equazione secolare dei due isotopi di uranio, che ne governano i decadimenti nel tempo e stimare attraverso il rapporto di queste il tempo al quale lo stesso assume il valore odierno fornito dal problema. 
		
		Per definizione il tempo di dimezzamento è il tempo medio al quale la metà di una quantità di materia dimezzi per effetto di decadimento. Tramite questo parametro è possibile scrivere una legge in forma di equazione differenziale che risolta fornirà l'equazione secolare. In particolare sia $ m(t)_{\chem{X}} $ la funzione descrivente la quantità di materia di una particolare specie chimica $ \chem{X} $. Per brevità di scrittura fino a quando il problema non imporrà di specificarla con gli isotopi dell'Uranio la specificazione di questa sarà omessa dai calcoli. Si ha che, se $ \tau_{\frac{1}{2}} $ è il tempo di dimezzamento, è possibile esprimere la vita media $ \tau $ e la costante di decadimento $ \lambda $ nel seguente modo:
		\begin{gather}
		\tau_{\frac{1}{2}}=\tau\ln(2) \implies \tau = \dfrac{\tau_{\frac{1}{2}}}{\ln(2)} \\
		\tau = \dfrac{1}{\lambda} \implies \lambda = \dfrac{1}{\tau}=\dfrac{\ln(2)}{\tau_{\frac{1}{2}}}
		\end{gather}
		La costante di decadimento a sua volta è interpretabile come la probabilità di decadimento per unità di tempo. Si ha dunque che, una quantità di materia $ m(t) $ in un'unità di tempo decrescerà di una quantità legata sia a $ \lambda $ che al proprio valore in quell'istante. Questo concetto può esprimersi matematicamente attraverso la seguente equazione differenziale:
		\begin{gather}
			-\der{m(t)}{t}=\lambda m(t)
		\end{gather}
		La soluzione di questa equazione differenziale è banale e nota, perciò non verrà riportata. Questa è:
		\begin{gather}
			m(t)=Ce^{-\lambda t}
		\end{gather}
		dove $ C $ è la costante d'integrazione. Questa rappresenta la quantità di materia presente all'istante zero, infatti $ m(0)=C $.
		
		I valori delle costanti di tempo e costante di decadimento calcolati dai dati del problema sono riportati nella tabella \ref{tbl:decayValues}.
		
		\begin{table}[h]
			\centering
			\caption{Costanti di tempo e di decadimento calcolate per l'uranio 235 e 238}
			\label{tbl:decayValues}
			\vspace{1mm}
		\begin{tabular}{c|cc}
			 
			& $^{238}U$ & $^{235}U$ \\ 
			\hline 
			$\tau$ & $\SI{6.49e9}{yr}$ & $\SI{1.03e9}{yr}$ \\ 
			 
			$\lambda$ & $\SI{1.54e-10}{yr^{-1}}$ & $\SI{9.71e-10}{yr^{-1}}$ \\ 
		\end{tabular} 
		\end{table}
		Applicando la teoria sviluppata al nostro problema, possiamo certamente dire che, sebbene il valore delle costanti di integrazione non sia noto, questo è uguale sotto la supposizione di uguale nucleogenesi. Dunque il rapporto tra le equazioni secolari è:
		\begin{gather}
		r(t)=\dfrac{m(t)_{^{235}\chem{U}}}{m(t)_{^{238}\chem{U}}}=e^{-(\lambda_{^{235}\chem{U}}-\lambda_{^{238}\chem{U}})t} \\
		\end{gather}
		Da cui, prendendo il logaritmo e dividendo per meno la differenza delle costanti di decadimento si ottiene:
		\begin{gather}
		\label{eqn:rappfin}
		t(r)=-\dfrac{\ln(r)}{(\lambda_{^{235}\chem{U}}-\lambda_{^{238}\chem{U}})}
		\end{gather}
		L'equazione \ref{eqn:rappfin} rappresenta il tempo $ t $ al quale il rapporto tra le quantità di isotopi di uranio è $ r $. Da cui, inserendo i dati forniti dal problema e quelli calcolati e riportati nella tabella $ \ref{tbl:decayValues} $ si ha:
		\begin{gather}
			t(r)=\SI{6.08e9}{yr}
		\end{gather}
		Questo valore sembra avere senso. Infatti è poco più grande del valore stimato dell'età della Terra: $ \SI{4.54e9}{yr} $.
	\end{svol}
\end{document}