\documentclass[../main.tex]{subfiles}

\begin{document}
	\begin{ese}[10.2]
		Le reazioni nucleari avvengono nella parte centrale del Sole:
		\begin{gather}
			R<0.2R_{\odot}
		\end{gather}
		In tale regione è contenuta circa un terzo della massa solare. Assumendo che il Sole sia composto totalmente da idrogeno, dare una stima della reattività:
		\begin{gather*}
			\left<\sigma\nu\right> 
		\end{gather*}
		per la reazione:
		\begin{gather}
		\label{eqn:fusHH}
			\chem{^{1}H}+\chem{^{1}H}=\chem{^{2}H}+e^{+}+\nu_{e}
		\end{gather}
	\end{ese}
	\begin{svol}
		La massa del Sole è:
		\begin{gather*}
		M=\SI{2e20}{\kilo\gram}
		\end{gather*}
		e la potenza irraggiata è di:
		\begin{gather*}
		L=\SI{4e26}{\watt}
		\end{gather*}
		Ogni processo di fusione libera $ \SI{24.68}{\mega\electronvolt}=	\SI{3.954e-12}{\joule} $. Dunque il tasso di fusioni è di:
		\begin{gather*}
		f=\dfrac{L}{\SI{3.954e-12}{\joule}}=\SI{4e38}{\per\second}
		\end{gather*}
		Per ogni fusione ci sono due reazioni del tipo \ref{eqn:fusHH} per cui il tasso di reazioni è di:
		\begin{gather}
		f_{H+H}=2f=\SI{8e38}{\per\second}
		\end{gather}
		La massa di un protone è di:
		\begin{gather*}
			m_{p}=\SI{938}{\mega\electronvolt\per c\squared}=\SI{1.673e-27}{\kilo\gram}
		\end{gather*}
		Per cui il numero di protoni presenti nella zona centrale del Sole è di:
		\begin{gather*}
		N_{H}=\dfrac{M_{\odot}}{3m_{p}}\approx \num{4e56}
		\end{gather*}
		Quindi la probabilità di fusione in un secondo è:
		\begin{gather*}
			\left<\sigma\nu\right>=\dfrac{f_{H+H}}{N_{H}}=\num{2e-18}
		\end{gather*}
	\end{svol}
\end{document}