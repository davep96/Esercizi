\documentclass[../main.tex]{subfile}

\begin{document}
	\begin{ese}[5.3]
		Usando le sezioni d'urto riportate nella tabella \ref{tbl:CrossSectionData}, ed assumendo che, se si usa acqua come moderatore, un neutrone abbia una probabilità del $47\%$ di essere catturato in acqua, cacolare la frazione di $ ^{235}\chem{U} $ richiesta per avere un reattore critico. Oggi la frazione di $ ^{235}\chem{U} $ è del $ 0.7\% $, ma in passato era maggiore: calcolare quando era sufficiente perché l'$ U $ naturale potesse essere critico.
		\begin{table}[]
			\centering
			\caption{Valori della sezione d'urto per il problema 5.3}
			\label{tbl:CrossSectionData}
			\begin{tabular}{c|cc}
				$\sigma [\si{\barn}]$ & $^{235}\chem{U}$
				 & $^{238}\chem{U}$    \\ \hline
				$\sigma_{tot}$                            & 703                                  & 12                  \\
				$\sigma_{fiss}$                           & 589                                  & $\SI{1.7e-5}{}$ \\
				$\sigma_{n,\gamma}$               & 99                                   & 2.7                
			\end{tabular}
		\end{table}
	\end{ese}
	\begin{svol}
		L'equazione che descrive il numero edio di neutroni prodotti per un sistema in cui avviene una reazione nucleare è:
		\begin{gather}
			N(t)=N(0)\exp\left[(\nu q-1)\dfrac{1}{\tau}\right]
		\end{gather}
		Dove $ N(t) $ è il numero di neutroni, $ \tau $ è il tempo medio necessario affinché un neutrone causi una fissione, $ \nu $ è il numero medio di neutroni prodotti da una fissione, e $ q $ è la probabilità che la fissione avvenga. Affiché il reattore possa autossotenersi è necessario che questo prodotto sia uguale ad uno, in modo tale da avere una soluzione \emph{critica}. Si suppone che la probabilità di fissione con l'uranio 238 possa essere trascurata. Come numero medio di neutroni prodotti per fissione, si utilizza il valore riportato a lezione di $ \nu\approx2.4 $. Sia $ x $ il rapporto tra gli isotopi di uranio. La sezione d'urto di fissione totale è quindi:
		\begin{gather}
			\overline{\sigma}_{fiss}=x\times\sigma_{fiss,^{235}\chem{U}}
		\end{gather}
		La sezione d'urto di cattura neutronicq con seguente emissione di fotone totale invece è:
		\begin{gather}
			\label{eqn:sigmafiss}
			\overline{\sigma}_{n,\gamma}=x\times\sigma_{n,\gamma,^{235}\chem{U}}+(1-x)\times\sigma_{n,\gamma,^{238}\chem{U}}
		\end{gather}
		Quindi, la probabilità di fissione causata da un elettrone non catuttato dall'acqua è:
		\begin{gather}
			p_{fiss}=\dfrac{\overline{\sigma}_{fiss}}{\overline{\sigma}_{fiss}+\overline{\sigma}_{n,\gamma}}=\dfrac{x\times\sigma_{fiss,^{235}\chem{U}}}{x\times\sigma_{fiss,^{235}\chem{U}}+x\times\sigma_{n,\gamma,^{235}\chem{U}}+(1-x)\times\sigma_{n,\gamma,^{238}\chem{U}}}
		\end{gather}
		Tuttavia non tutti i neutroni emessi colpiscono atomi di uranio, in parte ne vengono assorbiti dal moderatore, che in questo caso è l'acqua. La probabilità che vengano assorbiti, e quindi che non conseguano a reazioni nucleari è fornita dai dati. Si ha dunque che la probabilità totale di fissione è:
		\begin{gather}
			\overline{p}_{fiss}=53\%p_{fiss}
		\end{gather}
		Affinché la reazione sia in grado di autosostenersi è quindi necessario che:
		\begin{gather}
			\nu\overline{p}_{fiss}=1
		\end{gather}
		Invertendo le relazioni a partire dalla \ref{eqn:sigmafiss} si ottiene:
		\begin{gather}
		x\left(\left(\dfrac{53\nu-100}{100}\right)\sigma_{fiss}+\sigma_{n,\gamma,^{238}\chem{U}}-\sigma_{n,\gamma,^{235}\chem{U}}\right)=\sigma_{n,\gamma,^{238}\chem{U}}
		\end{gather}
		Da cui:
		\begin{gather}
		\label{eqn:Ufissperc}
		x=\dfrac{\sigma_{n,\gamma,^{238}\chem{U}}}{\left(\left(\dfrac{53\nu-100}{100}\right)\sigma_{fiss}+\sigma_{n,\gamma,^{238}\chem{U}}-\sigma_{n,\gamma,^{235}\chem{U}}\right)}=\dfrac{2.7}{63.908}=4.2\%
		\end{gather}
		Nell'esercizio \ref{ese:UDecay} era stata calcolata la relazione tra il tempo trascorso e il rapporto tra le quantità degli isotopi di Uranio presenti (l'equazione \ref{eqn:rappfin}). Inserendo il valore calcolato in \ref{eqn:Ufissperc} si trova:
		\begin{gather}
		t(4.2\%)=\SI{3.87e9}{yr}
		\end{gather}
		Purché questo risultato non sia identico a quello riportato alla fine dell'esercizio ne è una buona stima. Inffati l'errore presente è solo di un fattore due e può essere attribuito alle approssimazioni fatte nella modellizzazione matematica del problema.
	\end{svol}
\end{document}